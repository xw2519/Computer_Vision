\documentclass[9pt]{article}
\usepackage{multicol}
\usepackage{calc}
\usepackage{ifthen}
\usepackage{hyperref}
\usepackage[a4paper, total={7in, 10in}]{geometry}
\usepackage[parfill]{parskip}
\usepackage{graphicx}
\usepackage{multicol}
\usepackage{amsmath, xparse}

\setlength{\parindent}{0em}
\graphicspath{ {./Images/} }

\setlength{\columnsep}{1cm}
\title{Computer Vision Cheat Sheet}

\begin{document}
\maketitle

\begin{multicols}{2}
    \section{Image Filtering}
        \begin{enumerate}
        \item  
        \end{enumerate}

    \columnbreak 

    \section{Feature detection and description}
        \subsection{Harris Corner Detector}
        \begin{enumerate}
            \item Process: 
            \begin{itemize}
                \item Compute $x$ and $y$ derivatives of an image
                $$
                    I_x = G_x \ast I; \quad I_y = G_y \ast I
                $$
                where $G$ is a filter e.g. Sobel filter 

                \item At each pixel, compute the matrix $M$
                $$
                    M = \sum_{x,y} w(x,y) \begin{bmatrix}
                        I_x^2 & I_x I_y \\ 
                        I_x I_y & I_y^2
                    \end{bmatrix}
                $$

                \item Calculate detector response
                $$
                    R = \lambda_1 \lambda_2 - k(\lambda_1 + \lambda_2)^2
                $$

                \item Detect interest points which are local maxima and whose response $R$ are
                above a threshold
            \end{itemize}
        \end{enumerate}

        \subsection{SIFT}
        \begin{enumerate}
            \item Process:
            \begin{itemize}
                \item Detection of scale-space extrema 
                \item Keypoint localisation 
                \item Orientation assignment 
                \item Keypoint descriptor
            \end{itemize}
        \end{enumerate}

    


    

\end{multicols}

\pagebreak

\begin{multicols}{2}
    \section{Image classification, detection and segmentation}

    \columnbreak

    \section{Motion estimation and tracking}
    \subsection{Optic flow}
    \begin{enumerate}
        \item Optic flow assumptions:
        \begin{itemize}
            \item Brightness constancy: A pixel has constant brightness across time.
            \item Small motion: Between frames, motion is small.
            \item Spatial coherence: Pixels move like their neighbours i.e. flow is constant within a small neighbourhood.
        \end{itemize}

        \item Constraints:
        \begin{itemize}
            \item Brightness constancy assumption: 
            $$
                I(x + u,y + v,t + 1) = I(x,y,t)
            $$
            where $I$: Intensity, $(x,y,t)$: Spatial and temporal coordinates, $(u,v)$: Displacement

            \item Small motion assumption:
            $$
                I(x+u,y+v,t+1) \approx I(x,y,t) + \frac{\partial I}{\partial x}u + \frac{\partial I}{\partial y}v + \frac{\partial I}{\partial t}
            $$

            \item Optical flow constraint equation (Combining both): 
            $$
                \frac{\partial I}{\partial x}u + \frac{\partial I}{\partial y}v + \frac{\partial I}{\partial t} = 0
            $$
        \end{itemize}
    \end{enumerate}
\end{multicols}

\end{document}
