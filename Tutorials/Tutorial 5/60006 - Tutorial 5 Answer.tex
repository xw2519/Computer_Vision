\title{\vspace{-2.0cm} 60006 - Tutorial 5

Object Detection, Motion}
\author{
        Xin Wang
}
\date{\today}

\documentclass[12pt]{article}
\usepackage[margin=0.5in]{geometry} 
\usepackage{amsmath} 
\usepackage{graphicx}  
\usepackage[parfill]{parskip}

\setlength{\parindent}{0em}

\begin{document}
\maketitle

\section*{Question 1}

\textbf{Question 1.1:} Calculate the aspect ratio (width to height ratio) of the ROIs.

Aspect ratio:
\begin{itemize}
    \item A: $1:1$ 
    \item B: $2:1$
    \item C: $1:2$
\end{itemize}

\textbf{Question 1.2:}
\begin{enumerate}
    \item Count the number of ground truth persons in the figure.
    
    Number of ground truth person: $7$
    
    \item Calculate TP, FP, FN, precision and recall with a threshold of $0.9$.
    \begin{itemize}
        \item TP: 3 
        \item FP: 0 
        \item FN: 4
    \end{itemize}

    Precision: $1$

    Recall: $\frac{3}{7}$
\end{enumerate}

\textbf{Question 1.3:} Calculate TP, FP, FN, precision and recall using thresholds of 0.6, 0.7, 0.8, 0.9 and 0.95. Fill in
the following table.
\begin{table}[h]
    \centering
    \begin{tabular}{|c|c|c|c|c|c|}
    \hline
    Threshold & TP & FP & FN & Precision & Recall \\ \hline
    0.6       & 7  & 2  & 0  & 0.78      & 1      \\ \hline
    0.7       & 6  & 1  & 1  & 0.86      & 0.86   \\ \hline
    0.8       & 4  & 0  & 3  & 1         & 0.57   \\ \hline
    0.9       & 3  & 0  & 4  & 1         & 0.43   \\ \hline
    0.95      & 1  & 0  & 6  & 1         & 0.14   \\ \hline
    \end{tabular}
\end{table}

\pagebreak

\textbf{Question 1.4:} Plot the precision-recall curve using the previous table.



\textbf{Question 1.5:} Estimate the average precision using the precision-recall curve.

\section*{Question 2}

\textbf{Question 2.1:} Calculate the spatial and temporal image gradients at the shaded pixel using the finite
differences.
\begin{gather*}
    I_x = \frac{I(x+1, y, t) - I(x-1, y, t)}{2} = \frac{1}{2} \\ 
    I_y = \frac{I(x, y+1, t) - I(x, y-1, t)}{2} = \frac{1}{2} \\
    I_t = I(x,y,t+1) - I(x,y,t) = -1
\end{gather*}

\textbf{Question 2.2:} Write down the optic flow constraint equation at the shaded pixel.
\begin{gather*}
    I_x u + I_y v + I_t = 0 \\ 
    u + v = 0
\end{gather*}

\textbf{Question 2.3:} Can you solve the equation? If not, assume that the flow is constant with in $3 \times 3$
neighbourhood, add an additional equation at the striped pixel and solve the flow.

\section*{Question 3} 

Video denoising is a process to remove noise and other imaging artefacts such as scratches from
videos. Unlike single image denoising, where the only information available is in the current
picture, video denoising can borrow information from adjacent time frames. In order to do this
without introducing blur, video denoising requires accurate pixel-wise motion estimates. Describe
your idea of a possible video denoising algorithm. The input is a noisy video of a moving object
and the expected output is a clean video

\begin{enumerate}
    \item Estimate the optic flow between adjacent time frames using the Lucas-Kanade method.
    \item For each image to be denoised, warp its adjacent frames to this image according to the optic
    flow field
    \item Perform denoising by averaging the current frame with the warped adjacent frames
\end{enumerate}




\end{document}
