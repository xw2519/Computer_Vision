\title{\vspace{-2.0cm} 60006 - Tutorial 1

Image Formation, Image Filtering}
\author{
        Xin Wang
}
\date{\today}

\documentclass[12pt]{article}
\usepackage[margin=0.5in]{geometry} 
\usepackage{amsmath} 
\usepackage{graphicx}  
\usepackage[parfill]{parskip}

\setlength{\parindent}{0em}

\begin{document}
\maketitle

\section*{Question 1}
Let $(x, y)$ be the coordinate of a point in an image. Its homogeneous coordinate $(x, y, 1)$ is often
used in computer vision or graphics, for example, transforming an image by scaling, translation,
rotation etc. The transformation can be described by the following equation:
\begin{equation*}
    \begin{pmatrix}
        x' \\
        y' \\
        1 
    \end{pmatrix} = A . 
    \begin{pmatrix}
        x \\
        y \\
        1
    \end{pmatrix}
\end{equation*}
where $(x, y, 1)$ and $(x', y', 1)$ denote the coordinate before and after transformation, $A$ is a 3x3
transformation matrix. For the following examples of $A$, describe what kind of transformation it
performs.
\\
\\
\textbf{1.1: Example 1}: 
\begin{equation*}
    \begin{pmatrix}
        x' \\
        y' \\
        1 
    \end{pmatrix} = 
    \begin{pmatrix}
        1 & 0 & 10 \\
        0 & 1 & 5 \\
        0 & 0 & 1
    \end{pmatrix} . 
    \begin{pmatrix}
        x \\
        y \\
        1
    \end{pmatrix}
\end{equation*}

Calculating the following equation results in:
\begin{align*}
    \begin{pmatrix}
        x' \\
        y' \\
        1 
    \end{pmatrix} &= 
    \begin{pmatrix}
        1 & 0 & 10 \\
        0 & 1 & 5 \\
        0 & 0 & 1
    \end{pmatrix} . 
    \begin{pmatrix}
        x \\
        y \\
        1
    \end{pmatrix} \\ &=
    \left(\begin{matrix}
        1x+0y+10 \\
        0x+1y+5 \\
        0x+0y+1
    \end{matrix}\right) \\ &=
    \left(\begin{matrix}
        x+10 \\
        y+5 \\
        1
    \end{matrix}\right)
\end{align*}

This corresponds to a \textit{translation transformation} that shifts a pixel
$10$ in the $x$-axis and $5$ in the $y$-axis. 
\\
\\
\textbf{1.2: Example 2}: 
\begin{equation*}
    \begin{pmatrix}
        x' \\
        y' \\
        1 
    \end{pmatrix} = 
    \begin{pmatrix}
        5 & 0 & 0 \\
        0 & 5 & 0 \\
        0 & 0 & 5
    \end{pmatrix} . 
    \begin{pmatrix}
        x \\
        y \\
        1
    \end{pmatrix}
\end{equation*}

Calculating the following equation results in:
\begin{align*}
    \begin{pmatrix}
        x' \\
        y' \\
        1 
    \end{pmatrix} &= 
    \begin{pmatrix}
        5 & 0 & 0 \\
        0 & 5 & 0 \\
        0 & 0 & 5
    \end{pmatrix} . 
    \begin{pmatrix}
        x \\
        y \\
        1
    \end{pmatrix} \\ &=
    \left(\begin{matrix}
        5x+0y+0 \\
        0x+5y+0 \\
        0x+0y+5
    \end{matrix}\right) \\ &=
    \left(\begin{matrix}
        5x \\
        5y \\
        5
    \end{matrix}\right)
\end{align*}
This corresponds to a \textit{scaling transformation} that zooms the image
$5$ in the $x$-axis and $5$ in the $y$-axis i.e. zooming the image 5 times.

\section*{Question 2}
For a RGB image, at each pixel, there are intensity values for three channels (R: red; G: green; B:
blue). For example, $[255, 0, 0]$ represents pure red. $[0, 255, 0]$ represents pure green. $[0, 0, 255]$
represents pure blue. Each channel is represented by an integer between $0$ and $255$, i.e. an 8-bit
unsigned char.
\\
\\
\textbf{2.1:} For a RGB image of size $1280 \times 960$ pixels, without image compression, how many bytes are
needed for storing the image?
\\
\\
In a $1280 \times 960$ image, there are $1228800$ pixels. Each pixel needs to represent
a three channels which is represented by $8 \times 3$ bits. In totals there are:
\begin{align*}
    1228800 \text{ pixels} \times 24 &= 29491200 \text{ bits} \\ 
    &= 3686400 \text{ bytes}
\end{align*}

\textbf{2.2:} There are algorithms to convert a RGB image into a grayscale image, where each pixel only
has one intensity value, which represents the brightness. For example, one algorithm is
recommended by ITU-R Recommendation BT.601, which is formulated as the following
equation,
$$
    Y = 0.299R + 0.587G + 0.114B
$$
Could you work out what grayscale values pure red, pure green and pure blue respectively
convert to?
\\
\\ 
Pure Red: $\text{RGB} = [255, 0, 0]$ and grayscale is $[76.245, 0, 0]$
Pure Green: $\text{RGB} = [0, 255, 0]$ and grayscale is $[0, 149.685, 0]$
Pure Blue: $\text{RGB} = [0, 0, 255]$ and grayscale is $[0, 0, 29.07]$

\pagebreak

\section*{Question 3}

Suppose that we have an image that is corrupted by Gaussian white noise $Y = I +n$, where $I$
denotes the clean image, $n \sim N(0, \sigma^2)$ denotes the Gaussian white noise and $Y$ denotes the
corrupted image. 

Performing image filtering using a low-pass filter is one approach for
denoising, but it may also result in loss of fine details. Another approach is
to denoising is to take a lot of images of the same object and then combine them. For
example, you can take a lot of pictures of the moon in the sky from the same
angle. Each image is described by $Y_i = I + n_i(i = 1,2,..., N)$, where $n_i$ is one
independent realisation of the noise. Then you can take the average, $Y =
\frac{1}{N} \sum_{i=1}^N Y_i$. 

Please derive the mean and variance of the noise in the combined image $Y$.

The noise in the combined image $e$ is defined as:
\begin{align*}
    e &= Y - I \\
    &= \frac{1}{N} \sum_{i=1}^N n_i
\end{align*}

Since each independent realisation of noise is of type Gaussian white noise, the
mean of the noise in the comnbined image is: 
\begin{align*}
    E[e] &= \frac{1}{N}\sum_{i=1}^N E[n_i] \\
    &= 0
\end{align*}

The variance is calculated by:
\begin{align*}
    var[e] &= E[(e - E[e])^2] \\ 
    &= \frac{1}{N} \sigma^2
\end{align*}

\section*{Question 4}

Consider a 2D Gaussian filter:
$$
    h(x, y) = \frac{1}{2\pi \sigma^2} e^{-\frac{x^2 + y^2}{2\sigma^2}}
$$

\textbf{4.1:} Derive the first derivative: $\frac{\partial h}{\partial x}$ and
$\frac{\partial h}{\partial y}$

\begin{align*}
    \frac{\partial h}{\partial x} = -\dfrac{x\mathrm{e}^{-\frac{x^2+y^2}{2{\sigma}^2}}}{2{\pi}{\sigma}^4} \\
    \frac{\partial h}{\partial y} = -\dfrac{y\mathrm{e}^{-\frac{y^2+x^2}{2{\sigma}^2}}}{2{\pi}{\sigma}^4}
\end{align*}

\textbf{4.2:} The Gaussian filter has an infinitive support. People often
truncate and approximate the filter. 

Please check that whether the 2D Gaussian filter is equivalent o the convolution
of a 1D Gaussian filter along x-axis and a 1D Gaussian filter along y-axis. 

1D Gaussian filter along $x$-axis is $[1/6, 2/3, 1/6]$ and along $y$-axis is $\begin{bmatrix}
    1/6 \\ 2/3 \\ 1/6
\end{bmatrix}$. It is equivalent to the 2D Gaussian filter. 

\textbf{4.3:} What is the computational cost when we convolve an $N \times N$
image with a 3x3 2D Gaussian filter?
\begin{itemize}
    \item At each pixel, there are $K \times K$ (9) multiplications and $K^2 -
    1$ (8) summations
    \item In total, there are $N^2K^2$ multiplications and $N^2(K^2 - 1)$
    summations 
    \item Complexity is $O(N^2 K^2)$
\end{itemize}

\textbf{4.4:} What is the computational cost when we convolve an $N \times N$
image with two 1D Gaussian filters, respectively with size 1x3 and 3x1? 
\begin{itemize}
    \item At each pixel, there are $K$ multiplications and $K - 1$ summations
    \item In total, there are $2 N^2K$ multiplications and $2 N^2(K - 1)$
    summations 
    \item Complexity is $O(N^2 K)$
\end{itemize}

\textbf{4.5:} In general, what is the computational cost when we convolve an $N
\times N$ image with a $K \times K$ 2D Gaussian filter? Use the big O notation.
\begin{itemize}
    \item At each pixel, there are $K \times K$ multiplications and $K^2 - 1$ summations
    \item In total, there are $N^2K^2$ multiplications and $N^2(K^2 - 1)$
    summations 
    \item Complexity is $O(N^2 K^2)$
\end{itemize}

\end{document}

  