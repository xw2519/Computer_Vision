\documentclass{article} 

\usepackage[margin=0.5in]{geometry} 
\usepackage{amsmath} 
\usepackage{graphicx}  
\usepackage[parfill]{parskip}

\setlength{\parindent}{0cm}

\newcommand{\sol}{\textbf{Solution}:} 
\newcommand{\maketitletwo}[2][]{\begin{center}
        \Large{\textbf{Computer Vision #1 \\ } }
        \vspace{5pt}
        \normalsize{Xin Wang \\ 
        \today}   
        \vspace{15pt}
\end{center}}

\begin{document}
    \maketitletwo[Question Pack] 

    \section{Year 2018}
    \textbf{Question 1(b):} 
    \begin{enumerate}
        \item Explain how the Hough transform could be used to find these curve segments.

        This is possible using Generalised Hough Transform. As the curves are
        defined by the equation $y = ax^2 + bx + c$, the equation can be used to
        create a discrete voting space and perform voting in the parameter
        space. 
        
        \item  Discuss how to limit the voting space to 2D to solve the above problem. What are the main 
        advantages of using low dimensional voting spaces?

        Low dimensional voting spaces reduces the computational complexity
        required. This reduction in processing load means a faster feature
        extraction. 
    \end{enumerate}
    

    \textbf{Question 4(a):} Explain the term optical flow and the usual
    assumptions in using it for motion analysis in an image sequence.   
    \begin{itemize}
        \item Optical flow is the motion (flow) of brightness patterns (optic) in videos.
        \item Assumptions in optic flow:
        \begin{itemize}
            \item Brightness consistency: Pixel has constant brightness across
            time  
            \item Small motion: Between frames, motion is small 
            \item Spatial coherence: Pixels move like their neighbours  
        \end{itemize}
    \end{itemize}
    
    \section{Year 2017}
    \textbf{Question 1(c):} The features detected in the images will end up in a
    database used for object recognition.
    \begin{enumerate}
        \item What are the two main challenges in object recognition? Provide two examples of each challenge.
        
        \begin{itemize}
            \item Viewpoint variation: Objects viewed from different angles look
            very different. E.g. cakes look different from the top and side. 

            \item Occlusion: Objects can be obscured by other things that makes
            it difficult to process and identify these objects. E.g. a hand
            covering a cup.
        \end{itemize}

        \item List the 4 main steps for k-means clustering used in bag-of-features.
        \begin{itemize}
            \item Randomly initialize $K$ cluster centers
            \item Iterate until convergence:
            \item Assign each data point to the nearest center
            \item Recompute each cluster center as the mean of all points
            assigned to it
        \end{itemize}
    \end{enumerate}

    \textbf{Question 2}
    \begin{enumerate}
        \item \begin{itemize}
            \item  In the context of image sequence processing, what is feature and describe why local features are 
            desired for tracking?

            A feature is a piece of information that is relevant for solving a
            computational task related to a certain application e.g. points,
            edges and objects for computer vision. 

            Local features are features in specific locations of the image e.g.
            mountain peakss that are distinct and unique from any angles and brightness.

            \item What are feature descriptors and why are they used over the original features?
            
            Feature descriptors are interest points encoded into a series of
            numbers that acts as a numerical "fingerprint" to differentiate one
            feature from another.

            These are used over original features since feature descriptors have
            much lower dimensions that original images and, this reduction in
            dimensionality, reduces the processing overheads required.
        \end{itemize}

        \item \begin{itemize}
            \item What is meant by the following when measuring tracking errors: TP, RN, FP, TN?
            \begin{itemize}
                \item TP: True positve - Model correctly predicts a positive class 
                \item FP: False positive - Model incorrectly predicts a positive
                class 
                \item FN: False negative - Model incorrectly predicts a negative
                class  
                \item TN: True negative - Model correctly predicts a negative class 
            \end{itemize}

            \item How are precision and recall defined?
            \begin{itemize}
                \item Precision: Fraction of TP to sum of TP and FP
                \item Recall: Fraction of TP to sum of TP and FN
            \end{itemize}
        \end{itemize}
        
    \end{enumerate}


    
\end{document}