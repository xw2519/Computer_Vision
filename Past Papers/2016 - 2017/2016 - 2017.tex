\documentclass{article} 

\usepackage[margin=0.5in]{geometry} 
\usepackage{amsmath} 
\usepackage{graphicx}  
\usepackage[parfill]{parskip}

\setlength{\parindent}{0cm}

\newcommand{\sol}{\textbf{Solution}:} 
\newcommand{\maketitletwo}[2][]{\begin{center}
        \Large{\textbf{Computer Vision #1 \\ } }
        \vspace{5pt}
        \normalsize{Xin Wang \\ 
        \today}   
        \vspace{15pt}
\end{center}}

\begin{document}
    \maketitletwo[2019 - 2020] 

    \section{Question 2}
    \textbf{Question 2.1}: 
    \begin{enumerate}
        \item In the context of image sequence processing, what is a feature and describe
        why local features are desired for tracking?

        \begin{itemize}
            \item A feature is a piece of information relevant for solving a computational
            task related to a certain application e.g. points, edges or objects. 
            \item Local features (Keypoint features) are features that are in
            specific locations of the  images e.g. mountain peaks or building
            corners. These local features are unique and disinctive enough from
            any angle, lighting and scale. 
        \end{itemize}

        \item What are feature descriptors and why are they used over the original
        features?

        Feature descriptors are interesting information encoded into a series of
        numbers that acts as a numerical fingerprint to be used to differentiate
        one feature from another. 

        This is used over original features since feature descriptors are much
        lower dimensions than the original image and, this reduction in
        dimensionality, reduces the overheads in processing the images. 
    \end{enumerate}

    \textbf{Question 2.2}: 
    \begin{enumerate}
        \item What is meant by the following when measuring tracking errors?
        \begin{itemize}
            \item TP: True Positive - An outcome where the model correctly predicts the positive class
            \item TN: True Negative - An outcome where the model correctly predicts the negative class
            \item FN: False Negative - An outcome where the model incorrectly predicts the negative class
            \item FP: False Positive - An outcome where the model incorrectly predicts the positive class
        \end{itemize}

        \item How are precision and recall defined?
        \begin{itemize}
            \item Precision: Fraction of relevant instances among the retrieved instances
            \item Recall: Fraction of relevant instances that were retrieved
        \end{itemize}
    \end{enumerate}

    \textbf{Question 2.3}: Describe the steps of the Harris corner detector, providing the necessary
    equations in matrix form.

    The process of the harris corner detector: 
    \begin{enumerate}
        \item Compute $x$ and $y$ derivatives of an image
        $$
            I_x = G_x \ast I; \quad I_y = G_y \ast I
        $$
        where $G$ is a filter e.g. Sobel filter 

        \item At each pixel, compute the matrix $M$
        $$
            M = \sum_{x,y} w(x,y) \begin{bmatrix}
                I_x^2 & I_x I_y \\ 
                I_x I_y & I_y^2
            \end{bmatrix}
        $$

        \item Calculate detector response
        $$
            R = \lambda_1 \lambda_2 - k(\lambda_1 + \lambda_2)^2
        $$

        \item Detect interest points which are local maxima and whose response $R$ are
        above a threshold
    \end{enumerate}

    \textbf{Question 2.4}: Propose a suitable framework to track the balloons in this sequence, taking into
    consideration:
    \begin{itemize}
        \item What features you will use and how you will extract them.
        \item How you would distinguish between balloons and clouds
        \item A possible way to deal with a balloon disappearing from a few
        frames due to it being blocked by another. This problem is
        known as occlusion.
    \end{itemize}

    Possible features are patterns on the balloon, the balloon basket and the
    outline of the balloon. These can be extracted using the SIFT feature
    descriptor that takes in a $16 \times 16$ window to return a $4 \times 4$
    keypoint descriptor. 

    The balloons and clouds can be differentiated using shape and colour.

    To detect occlusion, cache the windows and detect if the windows are
    overlapping i.e. the balloons may block one another. If so, predict the
    motion of the balloon using Kalman filter. 
\end{document}