\documentclass{article} 

\usepackage{amsmath} 
\usepackage{enumitem}
\usepackage[margin=0.5in]{geometry} 
\usepackage{graphicx} 
\usepackage[parfill]{parskip}

\setlength{\parindent}{0cm}

\newcommand{\sol}{\textbf{Solution}:} 
\newcommand{\maketitletwo}[2][]{\begin{center}
        \Large{\textbf{Computer Vision Past Paper #1 \\ } }
        \vspace{5pt}
        \normalsize{Xin Wang \\ 
        \today}   
        \vspace{15pt}
\end{center}}

\begin{document}
    \maketitletwo[2019 - 2020] 

    \textbf{Question 1: Image Filtering}:
    \begin{enumerate}[label=\alph*)]
        \item  Prewitt filtering:
        \begin{enumerate}[label=(\roman*)]
            \item Write down the 3x3 horizontal Prewitt filter $h_x$ and vertical Prewitt filter $h_y$
            \begin{gather*}
                h_x = \begin{bmatrix}
                    1&0&-1 \\
                    1&0&-1 \\
                    1&0&-1 
                \end{bmatrix} \quad \text{ and } \quad 
                h_y = \begin{bmatrix}
                    1&1&1 \\
                    0&0&0 \\
                    -1&-1&-1 
                \end{bmatrix}
            \end{gather*}
            
            \item Perform convolution between the image and the filters. Write down the output.
            \begin{gather*}
                h_x = \begin{bmatrix}
                    -3 & -1 & 0 & 1 & 3 \\ 
                    -6 & -3 & 0 & 3 & 6 \\
                    -9 & -6 & 0 & 6 & 9 \\ 
                    -12 & -9 & 0 & 9 & 12 \\
                    -9 & -7 & 0 & 7 & 9
                \end{bmatrix} \quad \text{ and } \quad  
                h_y = \begin{bmatrix}
                    -3 & -5 & -6 & -5 & -3 \\
                    -2 & -4 & -6 & -4 & -2 \\
                    -2 & -4 & -6 & -4 & -2 \\ 
                    -2 & -4 & -6 & -4 & -2 \\
                    5 & 9 & 12 & 9 & 5
                \end{bmatrix}
            \end{gather*}

            \item Explain how the Prewitt filter $h_x$ can be separated as two filters.
            \begin{gather*}
                h_x = \begin{bmatrix}
                    1&0&-1 \\
                    1&0&-1 \\
                    1&0&-1 
                \end{bmatrix} = \begin{bmatrix}
                    1 \\ 1 \\ 1
                \end{bmatrix} * \begin{bmatrix}
                    1 & 0 & -1
                \end{bmatrix}
            \end{gather*}

            \item In certain computer vision tasks (e.g. Harris corner detection), to calculate the
            image gradient, Gaussian filtering is applied prior to Prewitt filtering. Explain the
            motivation for performing Gaussian filtering.

            Gaussian filtering is applied to a image would blur an image which
            removes details and noise. By operating similarly to the mean
            filter, this filter removes the noise which would affect the
            performance of the edge detection filters like Prewitt Filter. 
        \end{enumerate}

        \item In certain computer vision tasks (e.g. Harris corner detection), to calculate the
        image gradient, Gaussian filtering is applied prior to Prewitt filtering. Explain the
        motivation for performing Gaussian filtering.
        \begin{itemize}
            \item To implement the kernel, how would you design the kernel size $K$?
            \begin{gather*}
                [ -\sigma *k, \sigma *k ]
            \end{gather*}
            
            \item Suppose the Gaussian kernel size is $K \times K$ and the input
            image size is $N \times N$, evaluate the computational complexity using the big O notation for two
            implementations respectively: direct 2D Gaussian filtering and
            separable filtering.
            \begin{itemize}
                \item Direct 2D Gaussian
                \begin{gather*}
                    \text{Multiplications}: K^2 N^2 \\
                    \text{Summations}: N^2 (K^2 - 1) \\ 
                    O(K^2 N^2)
                \end{gather*}

                \item Separable 2D Gaussian
                \begin{gather*}
                    \text{Multiplications}: 2K N^2 \\
                    \text{Summations}: N^2 (2K - 1) \\ 
                    O(2K N^2)
                \end{gather*}
            \end{itemize}
        \end{itemize}
    \end{enumerate} 




    \textbf{Question 1B}: 
    
    ertert
\end{document}